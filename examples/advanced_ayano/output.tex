\documentclass[a4paper]{article}
\usepackage{graphicx, hyperref, xcolor, gensymb, amssymb, mathtools, wrapfig, mathtools, microtype, lastpage, caption, titlesec, paracol, longtable, booktabs, cancel}
\usepackage[T2A, T1]{fontenc}
\usepackage[utf8]{inputenc}
\usepackage[russian, ukrainian]{babel}
\hypersetup{
    colorlinks,
    linkcolor={blue!20!black},
    citecolor={blue!50!black},
    urlcolor={blue!80!black}
}

\usepackage[left = 25mm, right = 10mm, top=20mm, bottom=20mm, bindingoffset=0cm]{geometry}

% КОМАНДИ

\newcommand{\fnt}{
\fontshape{n}
\fontsize{14pt} {19pt}
\linespread{0.8}
\selectfont
} % нормативний шрифт
\newcommand{\capfnt}{
    \fnt
} % норматичний шрифт для підписів (тимчасово співпадає зі звичайним шрифтом)

\newcommand{\tb}{
    \hspace*{10mm}
} % відступ у 5 символів "x" відповідно до ДСТУ
\newcommand{\tbln}{
    \newline 
    \tb
} % те ж саме, але з переносом рядка
\newcommand{\tbsp}{
    \vspace*{1ex}
    \tbln
} % те ж саме, але з додатковим відступом між рядками

% бібліографія
\usepackage[square,sort,comma,numbers]{natbib}
\renewcommand{\bibsection}{}
\usepackage{totcount}
\newtotcounter{citnum} 
% лічильник цитувань
\def\oldbibitem{} \let\oldbibitem=\bibitem
\def\bibitem{\stepcounter{citnum}\oldbibitem}

% лічильник зображень
\newtotcounter{graphnum}
\def\oldincludegraphics{} \let\oldincludegraphics=\includegraphics
\def\includegraphics{\stepcounter{graphnum}\oldincludegraphics}

% лічильник таблиць (НЕАВТОМАТИЧНИЙ!!!)
\newtotcounter{tabnum}

% лічильник додатків (НЕАВТОМАТИЧНИЙ!!!)
\newtotcounter{dodnum}

% нумерація сторінок
\usepackage{fancyhdr}
\pagestyle{fancy}
\fancyhf{}
\renewcommand{\headrulewidth}{0pt}
\setlength{\headheight}{15.3pt}
\fancyhead[R]{\fnt \thepage}

% виноски
\renewcommand{\thefootnote}{\large\arabic{footnote})~}
\renewcommand{\footnoterule}{\rule{20mm}{0.4pt} \vspace*{0.5ex}}
% \newcommand{\vyn}[2]{
%     \footnote[#1]{\large #2}
% } % створити виноску

\let\oldFootnote\footnote
\renewcommand{\footnote}[1]{
    \oldFootnote{\large #1}
} % перегрузка виноски

\providecommand{\tightlist}{%
  \setlength{\itemsep}{0pt}\setlength{\parskip}{0pt}}

% підписи малюнків/таблиць
\captionsetup[figure]{name={\fnt Рисунок~},labelsep=period}
\captionsetup[table]{name={\fnt Таблиця~},labelsep=period}

% markdown strikeouts
\usepackage{soul}

% codeblocks
\usepackage{minted}

% enumeration
\usepackage[shortlabels]{enumitem}

\titleformat*{\section}{\fnt\LARGE\bfseries}
\titleformat*{\subsection}{\fnt\Large\bfseries}
\titleformat*{\subsubsection}{\fnt\Large}

\begin{document}
\thispagestyle{empty} % [A]:for cropping
% ------------
% source: https://tex.stackexchange.com/questions/173651/need-help-to-create-such-a-beautiful-title-page
\begingroup
\centering
\obeylines
\topskip 0pt % source: https://tex.stackexchange.com/questions/2326/vertically-center-text-on-a-page
\vspace*{\fill}
\textbf{
\LARGE 
\LARGE 
\vspace{100pt}
\LARGE 
\Large 
\Large 
\Large 
\huge 
\vspace{50pt}
\large 
\large 
\large 
\large 
\vspace*{\fill}
\hspace{\fill}\large  \rule{20mm}{0.4pt}
\vspace{10pt}
}
\endgroup
% ------------
\newpage
\tableofcontents

\newpage
\fnt
This is an example of advanced `Ayano` features.
\tbln First, there are insert blocks. These can load some code from specified python script, and insert it at the start:
\tbln Note that their displayed version at the end actually includes the loaded script. These blocks can be used for display too: \\
e = 2.7182818284590455, pi = 3.141592653589793, pi != 2.7182818284590455
\tbln Sometimes you wish to only include part of the script, and do something else after some point. To achieve that, you can include `# TOSHINO KYOKO!` comment (note the space and casing; you can append anything after it at that line). Once this comment is seen by `Ayano`, it stops inserting the file, and appends the actual code you've provided in the block (possibly none). Comment itself is not included: 2
\tbln Ayano blocks can actually interact with the filesystem themselves. The caveat is - where exactly python thinks it's current directory is.
\tbln Well, the rule of thumb turns out to be quite simple:
\begin{itemize}
\item if there is an insertion declared for a block, it's execution will be in the directory that file is in
\item if there is no insertion declared, python will continue it's execution at `kyomato`'s process directory
\end{itemize}5
\tbln Ayano can be used to insert csv tables (possibly generated at runtime too). Here's a special syntax for it:
\begin{table}[h!]
\fnt
\begin{center}
\begin{tabular}{|c|c|c|c|c|}\hline
First Name & Last Name & Age & Salary & Tenacity\\ \hline
\hline
Wilson & Bennett & 19 & 5639 & 97155\\ \hline
Hailey & Robinson & 21 & 8419 & 68689\\ \hline
Edith & Morris & 24 & 8134 & 10937\\ \hline
Adison & Brown & 20 & 6131 & 41745\\ \hline
Vivian & Cunningham & 22 & 7834 & 35514\\ \hline
Heather & Roberts & 24 & 4881 & 73425\\ \hline
Michael & Armstrong & 23 & 5781 & 30899\\ \hline
Miranda & Martin & 19 & 8283 & 33619\\ \hline
Alexander & Kelley & 29 & 1990 & 98055\\ \hline
Chloe & Spencer & 24 & 779 & 89818\\ \hline
Sam & Johnston & 25 & 1147 & 71225\\ \hline
Alissa & Spencer & 18 & 3877 & 64965\\ \hline
Abraham & Kelley & 20 & 1622 & 65308\\ \hline
Edwin & Stewart & 20 & 4376 & 01364\\ \hline
Adam & Grant & 19 & 9616 & 38385\\ \hline
\end{tabular}
\stepcounter{tabnum}
\caption{\capfnt cred: \href{http://randat.com/}{randat}}
\label{tab:random_table1}
\end{center}
\end{table}

\tb This can be combined with insertion blocks to achieve generated table insertion; you can specify a path relative to the inserted file,
\begin{table}[h!]
\fnt
\begin{center}
\begin{tabular}{|c|c|c|c|}\hline
name & value & err & \%\\ \hline
\hline
tau & 4.3535223 & 0.0034342 & 8e-06\\ \hline
pi & 4334.34434 & 10 & 2.3e-05\\ \hline
chi & -2332.332424 & 1e-07 & -0.0\\ \hline
zeta & 4444.0 & 0.033 & 0.0\\ \hline
\end{tabular}
\stepcounter{tabnum}\end{center}
\end{table}

\tb ... but you've not forced to do that; you may do execution-relative path too:
\begin{table}[h!]
\fnt
\begin{center}
\begin{tabular}{|c|c|c|c|}\hline
name & value & err & \%\\ \hline
\hline
tau & 4.3535223 & 0.0034342 & 8e-06\\ \hline
pi & 4334.34434 & 10 & 2.3e-05\\ \hline
chi & -2332.332424 & 1e-07 & -0.0\\ \hline
zeta & 4444.0 & 0.033 & 0.0\\ \hline
\end{tabular}
\stepcounter{tabnum}\end{center}
\end{table}

\tb An of course, once data was already generated (it happened twice here, twice, in fact), you can insert it without any script:
\begin{table}[h!]
\fnt
\begin{center}
\begin{tabular}{|c|c|c|c|}\hline
name & value & err & \%\\ \hline
\hline
tau & 4.3535223 & 0.0034342 & 8e-06\\ \hline
pi & 4334.34434 & 10 & 2.3e-05\\ \hline
chi & -2332.332424 & 1e-07 & -0.0\\ \hline
zeta & 4444.0 & 0.033 & 0.0\\ \hline
\end{tabular}
\stepcounter{tabnum}\end{center}
\end{table}

\tb You can also restrict the columns and rows you want, for example:
\begin{table}[h!]
\fnt
\begin{center}
\begin{tabular}{|c|}\hline
name\\ \hline
\hline
chi\\ \hline
\end{tabular}
\stepcounter{tabnum}
\caption{\capfnt That's a-bit-cut table!}
\end{center}
\end{table}

\tb There's also a special columns syntax to display some columns in value-error format:
\begin{table}[h!]
\fnt
\begin{center}
\begin{tabular}{|c|c|c|}\hline
name & value & \%\\ \hline
\hline
tau & 4.354$\pm$0.003 & 8e-06\\ \hline
pi & 4334$\pm$10 & 2.3e-05\\ \hline
chi & -2332.33242400$\pm$0.00000010 & -0.0\\ \hline
zeta & 4444.00$\pm$0.03 & 0.0\\ \hline
\end{tabular}
\stepcounter{tabnum}
\caption{\capfnt That's a table with value-error formatting!}
\end{center}
\end{table}

\tb That's pretty much it, regarding tables.
\tbln Ayano also allows you to include dynamically-generated figures. You can go two ways about that:
\subsubsection{1st way}

\tb Generate a figure in a static block, and save it somewhere. After that, include it as an image:
\begin{figure}[h!]
\centering
\includegraphics[width=0.9\textwidth]{./sine_graph.png}
\caption{\capfnt that's a nice little sine graph!}
\label{fig:sine}
\end{figure}

\subsubsection{2nd way (more idiomatic, in my opinion)}

\tb Generate a figure in function block, and insert it with a special syntax:
\begin{figure}[h!]
\centering
\includegraphics[width=0.9\textwidth]{././sine_graph.png}
\caption{\capfnt That's another way to include a figure from Ayano!}
\label{fig:other_sine_graph}
\end{figure}

\tb The only real upside I can see here - you can generate-and-include figures with prior-unknown names or locations
\tbln Oh, and you also can specify width of the figure this way, if it matters to you, that is
\tbln And of course you can generate a figure with inserted block too!
\begin{figure}[h!]
\centering
\includegraphics[width=0.2\textwidth]{./source/other_sine_graph.png}
\caption{\capfnt The same sine. Again.}
\label{fig:yet_another_sine_graph}
\end{figure}

\tb Note, that in the last example, script actually saves sine into a folder near input markdown fine, and yet `Kyomato` will be able to find it (if configured properly, that is)
\subsection{Code blocks}

\begin{minted}[linenos, mathescape, autogobble, breaklines]{python}
e = 1.0
f = 1.0
for i in range(1, 100):
    f /= i
    e += f
not_pi = e
\end{minted}

That's a way to calculate pi \\

\begin{minted}[linenos, mathescape, autogobble, breaklines]{python}
e = 1.0
f = 1.0
for i in range(1, 100):
    f /= i
    e += f
from math import pi
f"e = {e}, pi = {pi}, pi != {not_pi}"
\end{minted}

Ooops, that actually turns out to be a formula for e :( \\

\begin{minted}[linenos, mathescape, autogobble, breaklines]{python}
# You will see this comment
x = 2
# And you'll see this one
# you'll see this comment, and the 2
x
\end{minted}

the all-mighty \textbf{2}!!! \\

\begin{minted}[linenos, mathescape, autogobble, breaklines]{python}
three = None
with open("3", "r") as f:
    three = int(f.readline())

\end{minted}

Here you can see an empty static block displayed as inserted script \\

\begin{minted}[linenos, mathescape, autogobble, breaklines]{python}
two = None
with open("2", "r") as f:
    two = int(f.readline())
\end{minted}

You can clearly see that this block has no insert statement, yet it knows \textbf{exactly} where to find the file \\

\begin{minted}[linenos, mathescape, autogobble, breaklines]{python}
two + three
\end{minted}

This block here uses variables prior-defined in static blocks. \\

\begin{minted}[linenos, mathescape, autogobble, breaklines]{python}
@csv_table: src=data/20240327230336_8798.csv, caption="cred: [randat](http://randat.com/)", ident = "random_table1"
\end{minted}

table insertion example \\

\begin{minted}[linenos, mathescape, autogobble, breaklines]{python}
import csv

# let's just create a bunch of variables and fill a local generated table with them
with open("calculations_table.csv", "w") as f:
    writer = csv.DictWriter(
        f,
        fieldnames=[
            "name",
            "value",
            "err",
            "%",
        ],
        quoting=csv.QUOTE_NONNUMERIC,
        delimiter=" ",
        quotechar="'",
    )
    writer.writeheader()

    tau = (4.3535223, 0.0034342)
    pi = (4334.34434, 10)
    chi = (-2332.332424, 1e-7)
    zeta = (4444.00, 0.033)

    def write_var(writer, val, name):
        writer.writerow(
            {
                "name": name,
                "value": val[0],
                "err": val[1],
                "%": round(val[1] / val[0] / 100, ndigits=6),
            }
        )

    write_var(writer, tau, 'tau')
    write_var(writer, pi, 'pi')
    write_var(writer, chi, 'chi')
    write_var(writer, zeta, 'zeta')

# say, we do something undesirable in this script next, and we do not want kyomato to execute that
@csv_table: src=calculations_table.csv
\end{minted}

An example of runtime-generated table insertion \\

\begin{minted}[linenos, mathescape, autogobble, breaklines]{python}
import csv

# let's just create a bunch of variables and fill a local generated table with them
with open("calculations_table.csv", "w") as f:
    writer = csv.DictWriter(
        f,
        fieldnames=[
            "name",
            "value",
            "err",
            "%",
        ],
        quoting=csv.QUOTE_NONNUMERIC,
        delimiter=" ",
        quotechar="'",
    )
    writer.writeheader()

    tau = (4.3535223, 0.0034342)
    pi = (4334.34434, 10)
    chi = (-2332.332424, 1e-7)
    zeta = (4444.00, 0.033)

    def write_var(writer, val, name):
        writer.writerow(
            {
                "name": name,
                "value": val[0],
                "err": val[1],
                "%": round(val[1] / val[0] / 100, ndigits=6),
            }
        )

    write_var(writer, tau, 'tau')
    write_var(writer, pi, 'pi')
    write_var(writer, chi, 'chi')
    write_var(writer, zeta, 'zeta')

# say, we do something undesirable in this script next, and we do not want kyomato to execute that
@csv_table: src=data/calculations_table.csv
\end{minted}

An example of runtime-generated table insertion with execution-relative path \\

\begin{minted}[linenos, mathescape, autogobble, breaklines]{python}
@csv_table: src=data/calculations_table.csv
\end{minted}

An example of runtime-generated table insertion with execution-relative path \\

\begin{minted}[linenos, mathescape, autogobble, breaklines]{python}
@csv_table: src=data/calculations_table.csv, rows = 3..4, columns = ["name"], caption = "That's a-bit-cut table!"
\end{minted}

This is only a part of the table above, due to specified column and row restrictions \\

\begin{minted}[linenos, mathescape, autogobble, breaklines]{python}
@csv_table: src=data/calculations_table.csv, columns = ["name", ("value", "err"), "%"], caption = "That's a table with value-error formatting!"
\end{minted}

You can see values having error right beside them, and formatted accordingly! \\

\begin{minted}[linenos, mathescape, autogobble, breaklines]{python}
import numpy as np
import matplotlib.pyplot as plt
from math import pi
x = np.arange(-pi, pi, 0.01)
y = np.sin(x)
plt.plot(x, y)
plt.savefig("sine_graph.png")
\end{minted}

some figure generation \\

\begin{minted}[linenos, mathescape, autogobble, breaklines]{python}
import numpy as np
import matplotlib.pyplot as plt
from math import pi
x = np.arange(-pi, pi, 0.01)
y = np.sin(x)
plt.plot(x, y)
plt.savefig("sine_graph.png")
@fig: src = sine_graph.png, ident = "other_sine_graph", caption = "That's another way to include a figure from Ayano!", width = 0.9
\end{minted}

that's a figure insertion syntax, all-in-one! \\

\begin{minted}[linenos, mathescape, autogobble, breaklines]{python}
import numpy as np
import matplotlib.pyplot as plt
from math import pi
x = np.arange(-pi, pi, 0.01)
y = np.sin(x)
plt.plot(x, y)
plt.savefig("../source/other_sine_graph.png")

# let's say, that there's some debugging going on here:
@fig: src = other_sine_graph.png, ident = "yet_another_sine_graph", caption = "The same sine. Again.", width = 0.2
\end{minted}

That's a third way to do the same thing! \\


\end{document}