\documentclass[a4paper]{article}
\usepackage{graphicx, hyperref, xcolor, gensymb, amssymb, mathtools, wrapfig, mathtools, microtype, lastpage, caption, titlesec, paracol, longtable, booktabs, cancel}
\usepackage[T2A, T1]{fontenc}
\usepackage[utf8]{inputenc}
\usepackage[russian, ukrainian]{babel}
\hypersetup{
    colorlinks,
    linkcolor={blue!20!black},
    citecolor={blue!50!black},
    urlcolor={blue!80!black}
}

\usepackage[left = 25mm, right = 10mm, top=20mm, bottom=20mm, bindingoffset=0cm]{geometry}

% КОМАНДИ

\newcommand{\fnt}{
\fontshape{n}
\fontsize{14pt} {19pt}
\linespread{0.8}
\selectfont
} % нормативний шрифт
\newcommand{\capfnt}{
    \fnt
} % норматичний шрифт для підписів (тимчасово співпадає зі звичайним шрифтом)

\newcommand{\tb}{
    \hspace*{10mm}
} % відступ у 5 символів "x" відповідно до ДСТУ
\newcommand{\tbln}{
    \newline 
    \tb
} % те ж саме, але з переносом рядка
\newcommand{\tbsp}{
    \vspace*{1ex}
    \tbln
} % те ж саме, але з додатковим відступом між рядками

% бібліографія
\usepackage[square,sort,comma,numbers]{natbib}
\renewcommand{\bibsection}{}
\usepackage{totcount}
\newtotcounter{citnum} 
% лічильник цитувань
\def\oldbibitem{} \let\oldbibitem=\bibitem
\def\bibitem{\stepcounter{citnum}\oldbibitem}

% лічильник зображень
\newtotcounter{graphnum}
\def\oldincludegraphics{} \let\oldincludegraphics=\includegraphics
\def\includegraphics{\stepcounter{graphnum}\oldincludegraphics}

% лічильник таблиць (НЕАВТОМАТИЧНИЙ!!!)
\newtotcounter{tabnum}

% лічильник додатків (НЕАВТОМАТИЧНИЙ!!!)
\newtotcounter{dodnum}

% нумерація сторінок
\usepackage{fancyhdr}
\pagestyle{fancy}
\fancyhf{}
\renewcommand{\headrulewidth}{0pt}
\setlength{\headheight}{15.3pt}
\fancyhead[R]{\fnt \thepage}

% виноски
\renewcommand{\thefootnote}{\large\arabic{footnote})~}
\renewcommand{\footnoterule}{\rule{20mm}{0.4pt} \vspace*{0.5ex}}
% \newcommand{\vyn}[2]{
%     \footnote[#1]{\large #2}
% } % створити виноску

\let\oldFootnote\footnote
\renewcommand{\footnote}[1]{
    \oldFootnote{\large #1}
} % перегрузка виноски

\providecommand{\tightlist}{%
  \setlength{\itemsep}{0pt}\setlength{\parskip}{0pt}}

% підписи малюнків/таблиць
\captionsetup[figure]{name={\fnt Рисунок~},labelsep=period}
\captionsetup[table]{name={\fnt Таблиця~},labelsep=period}

% markdown strikeouts
\usepackage{soul}

% codeblocks
\usepackage{minted}

% enumeration
\usepackage[shortlabels]{enumitem}

\titleformat*{\section}{\fnt\LARGE\bfseries}
\titleformat*{\subsection}{\fnt\Large\bfseries}
\titleformat*{\subsubsection}{\fnt\Large}

\begin{document}
\thispagestyle{empty} % [A]:for cropping
% ------------
% source: https://tex.stackexchange.com/questions/173651/need-help-to-create-such-a-beautiful-title-page
\begingroup
\centering
\obeylines
\topskip 0pt % source: https://tex.stackexchange.com/questions/2326/vertically-center-text-on-a-page
\vspace*{\fill}
\textbf{
\LARGE 
\LARGE 
\vspace{100pt}
\LARGE 
\Large 
\Large 
\Large 
\huge 
\vspace{50pt}
\large 
\large 
\large 
\large 
\vspace*{\fill}
\hspace{\fill}\large  \rule{20mm}{0.4pt}
\vspace{10pt}
}
\endgroup
% ------------
\newpage
\tableofcontents

\newpage
\fnt
Here's a basic Ayano explanation:
\tbln Essentially, Ayano blocks are regular code blocks, having `python, Ayano` marked as their language. This sort of codeblocks are treated completely differently by the parser - these blocks are supposed to allow arbitrary python code execution for data generation and analysis.
\tbln For this example, we'll only consider examples that do not interact with the filesystem.
\tbln Say, you have some sort of dataset, you wish to find a mean a deviation of. Here's a way to do that with Ayano (without filesystem interaction):
\tbln You might notice, that there's actually nothing displayed above in the document. That's due to \textit{static blocks} leaving nothing behind them. To actually display the dataset, we would use a regular, non-static Ayano block: \newline
[1, 1, 2, 3, 2, 5, 5, 6, 2, 1, 4, 9]
\tbln Above you should see a python's `str` representation of initial dataset.
\tbln Now, let's try calculating mean and the deviation. To achieve that, I'll use `statistics` module: 3.4$\pm$2.5
\tbln You should see dataset's mean value and deviation above. You might notice, that's it's in a good-looking format. That's because of special format used for value-deviation sort-of-values, that's described in the docs. If you wish to circumvent it, you absolutely can: 3.4166666666666665 $\pm$ 2.4664414311581235
\tbln Ayano can generate tables too! For example, let's generate a table of our dataset, but also add information on if it's greater or less than the mean value, it's square and absolute deviation from the mean:
\begin{table}[h!]
\fnt
\begin{center}
\begin{tabular}{|c|c|c|c|}\hline
initial data & cmp to mean & square & absdev\\ \hline
\hline
1.0$\pm$1.0 & less & 1 & 2.4166666666666665\\ \hline
1.0$\pm$1.0 & less & 1 & 2.4166666666666665\\ \hline
2.0$\pm$1.4 & less & 4 & 1.4166666666666665\\ \hline
3.0$\pm$1.7 & less & 9 & 0.4166666666666665\\ \hline
2.0$\pm$1.4 & less & 4 & 1.4166666666666665\\ \hline
5.0$\pm$2.2 & greater & 25 & 1.5833333333333335\\ \hline
5.0$\pm$2.2 & greater & 25 & 1.5833333333333335\\ \hline
6.0$\pm$2.4 & greater & 36 & 2.5833333333333335\\ \hline
2.0$\pm$1.4 & less & 4 & 1.4166666666666665\\ \hline
1.0$\pm$1.0 & less & 1 & 2.4166666666666665\\ \hline
4.0$\pm$2.0 & greater & 16 & 0.5833333333333335\\ \hline
9$\pm$3 & greater & 81 & 5.583333333333334\\ \hline
\end{tabular}
\stepcounter{tabnum}
\caption{\capfnt A crazy table containing a bunch of stuff}
\label{tab:crazy_table}
\end{center}
\end{table}

\tbln Above you can see that crazy table, and you also can refer to it (\ref{tab:crazy_table}).
\tbln That's pretty much it for Ayano blocks that do not access the filesystem.
\subsection{Code blocks}

\begin{minted}[linenos, mathescape, autogobble, breaklines]{python}
# first, you declare the dataset itself in a *static* Ayano block
# (note the exclamation mark above)
data = [1, 1, 2, 3, 2, 5, 5, 6, 2, 1, 4, 9]
# Static block essentially translated into a code initially executed by python, when the entire module it loaded.
# Keep that in mind: ALL changes introduced by static blocks will be visible by ALL function blocks.
\end{minted}

That's a static block, declaring initial dataset
\newline

\begin{minted}[linenos, mathescape, autogobble, breaklines]{python}
from statistics import stdev, mean
mean_val = mean(data)
deviation = stdev(data)
@dev: mean_val, deviation
\end{minted}

mean and deviation calculations
\newline

\begin{minted}[linenos, mathescape, autogobble, breaklines]{python}
from statistics import stdev, mean
mean_val = mean(data)
deviation = stdev(data)
f"{mean_val} $\pm$ {deviation}"
\end{minted}

mean and deviation calculations, but without value-error formatting
\newline

\begin{minted}[linenos, mathescape, autogobble, breaklines]{python}
# look at the last line first
# that's a table generation syntax - it involves a cell generator, number of rows, number of columns and optional ident and caption arguments
# you could use python's function as generator too (a general requirement is for the thing to be callable with two integer args)
from statistics import mean
from math import sqrt
header = ["initial data", "cmp to mean", "square", "absdev"]
mean_val = mean(data)
cells = list([[("err", val, sqrt(val)), "greater" if val >= mean_val else "less", val * val, abs(val - mean_val)] for val in data])
@gen_table: lambda r,c: header[c] if r == 0 else cells[r-1][c]; rows = 13, columns = 4, ident="crazy_table", caption = "A crazy table containing a bunch of stuff"
\end{minted}

that chaotic table generation example
\newline


\end{document}