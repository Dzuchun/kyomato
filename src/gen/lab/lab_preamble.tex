\documentclass[a4paper]{article}
\usepackage{graphicx, hyperref, xcolor, gensymb, amssymb, mathtools, wrapfig, mathtools, microtype, lastpage, caption, titlesec, paracol, longtable, booktabs, cancel, soul}
\usepackage[T2A, T1]{fontenc}
\usepackage[utf8]{inputenc}
\usepackage[russian, ukrainian]{babel}
\hypersetup{
    colorlinks,
    linkcolor={blue!20!black},
    citecolor={blue!50!black},
    urlcolor={blue!80!black}
}

\usepackage[left = 25mm, right = 10mm, top=20mm, bottom=20mm, bindingoffset=0cm]{geometry}

% КОМАНДИ

\newcommand{\fnt}{
\fontshape{n}
\fontsize{14pt} {19pt}
\linespread{0.8}
\selectfont
} % нормативний шрифт
\let\oldCaption\caption
\renewcommand{\caption}[1]{
\oldCaption{\fnt #1}
} % підпис нормативним шрифтом.

\newcommand{\tb}{
    \hspace*{10mm}
} % відступ у 5 символів "x" відповідно до ДСТУ
\newcommand{\tbln}{
    \newline 
    \hspace*{10mm}
} % те ж саме, але з переносом рядка
\newcommand{\tbsp}{
    \vspace*{1ex}
    \newline
    \hspace*{10mm}
} % те ж саме, але з додатковим відступом між рядками

\iffalse
\let\oldSection\section
\renewcommand{\section}[1]{
    \oldSection*{\tb #1}
    \addcontentsline{toc}{section}{#1}
    \fnt
    \tb
} % відображає у тексті та змісті наданий заголовок
\fi

\iftrue
\let\oldSubSection\subsection
\renewcommand{\subsection}[1]{
    \oldSubSection*{\tb #1}
    \addcontentsline{toc}{subsection}{#1}
    \fnt
    \tb
} % відображає у тексті та змісті наданий підзаголовок
\fi

% бібліографія
\usepackage[square,sort,comma,numbers]{natbib}
\renewcommand{\bibsection}{}
\usepackage{totcount}
\newtotcounter{citnum} 
% лічильник цитувань
\def\oldbibitem{} \let\oldbibitem=\bibitem
\def\bibitem{\stepcounter{citnum}\oldbibitem}

% лічильник зображень
\newtotcounter{graphnum}
\def\oldincludegraphics{} \let\oldincludegraphics=\includegraphics
\def\includegraphics{\stepcounter{graphnum}\oldincludegraphics}

% лічильник таблиць (НЕАВТОМАТИЧНИЙ!!!)
\newtotcounter{tabnum}

% лічильник додатків (НЕАВТОМАТИЧНИЙ!!!)
\newtotcounter{dodnum}

% нумерація сторінок
\usepackage{fancyhdr}
\pagestyle{fancy}
\fancyhf{}
\renewcommand{\headrulewidth}{0pt}
\setlength{\headheight}{15.3pt}
\fancyhead[R]{\fnt \thepage}

% виноски
\renewcommand{\thefootnote}{\large\arabic{footnote})~}
\renewcommand{\footnoterule}{\rule{20mm}{0.4pt} \vspace*{0.5ex}}
% \newcommand{\vyn}[2]{
%     \footnote[#1]{\large #2}
% } % створити виноску

\let\oldFootnote\footnote
\renewcommand{\footnote}[1]{
    \oldFootnote{\large #1}
} % перегрузка виноски

\providecommand{\tightlist}{%
  \setlength{\itemsep}{0pt}\setlength{\parskip}{0pt}}

% підписи малюнків/таблиць
\captionsetup[figure]{name={\fnt Рисунок~},labelsep=period}
\captionsetup[table]{name={\fnt Таблиця~},labelsep=period}

% idk why, but this redefinition does not work, for some reason
% \let\oldCaption\caption
% \renewcommand{\caption}[1] {
%     \oldCaption{\fnt #1}
% }